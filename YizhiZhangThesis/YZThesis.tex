
%%%%%%%%%%%%%%%%%%%%%%%%%%%%%%%%%%%%%%%%%%%%%%%%%%%%%%%

\documentclass{iitthesis}
%\documentclass[draft]{iitthesis}

% Document Options:
%
% Note if you want to save paper when printing drafts,
% replace the above line by
%
%   \documentclass[draft]{iitthesis}
%
% See Help file for more about options.

\usepackage[dvips]{graphicx}    % This package is used for Figures
\usepackage{rotating}           % This package is used for landscape mode.
\usepackage{epsfig}
\usepackage{subfigure}          % These two packages, epsfig and subfigure, are used for creating subplots.
% Packages are explained in the Help document.
\usepackage{amsmath,amssymb,amsthm,mathtools,bbm,booktabs,array,tikz,pifont,comment,multirow,url,graphicx}
\usepackage{color}
\input FJHDef.tex

%Requires ApproxUnivariate_k.tex, univariate_integration_k.tex, ConesPaperSpikyquad.eps, ConesPaperFlukyquad.eps

\DeclareMathOperator{\Var}{Var}
\DeclareMathOperator{\INT}{INT}
\DeclareMathOperator{\APP}{APP}
\DeclareMathOperator{\lin}{lin}
\DeclareMathOperator{\up}{up}
\DeclareMathOperator{\lo}{lo}
\DeclareMathOperator{\fix}{fix}
\DeclareMathOperator{\err}{err}
\DeclareMathOperator{\maxcost}{maxcost}
\DeclareMathOperator{\mincost}{mincost}
\newcommand{\herr}{\widehat{\err}}

\newtheorem{theorem}{Theorem}
\newtheorem{prop}[theorem]{Proposition}
\newtheorem{lem}{Lemma}
\newtheorem{cor}{Corollary}
\theoremstyle{definition}
\newtheorem{algo}{Algorithm}
\newtheorem{condit}{Condition}
%\newtheorem{assump}{Assumption}
\theoremstyle{remark}
\newtheorem{rem}{Remark}
\newcommand{\Fnorm}[1]{\abs{#1}_{\cf}}
\newcommand{\Ftnorm}[1]{\abs{#1}_{\tcf}}
\newcommand{\Gnorm}[1]{\norm[\cg]{#1}}
\newcommand{\flin}{f_{\text{\rm{lin}}}}

\begin{document}

%%% Declarations for Title Page %%%
\title{Title}
\author{Yizhi Zhang}
\degree{Doctor of Philosophy}
\dept{Applied Mathematics}
\date{Date}
\copyrightnoticetrue      % crate copyright page or not
%\coadvisortrue           % add co-advisor. activate it by removing % symbol to add co-advisor
\maketitle                % create title and copyright pages


\prelimpages         % Settings of preliminary pages are done with \prelimpages command


%%%  Acknowledgement %%%
\begin{acknowledgement}     % acknowledgement environment, this is optional
\par  Will be added once thesis is finished
% or \input{acknowledgement.tex} % you need a separate acknowledgement.tex file to include it.
\end{acknowledgement}


% Table of Contents
\tableofcontents
\clearpage

% List of Tables
\listoftables

\clearpage

%List of Figures
\listoffigures

\clearpage

%List of Symbols(optional)

\listofsymbols
 \SymbolDefinition{$\beta$}{List of symbols will be added later}

 \clearpage



%%% Abstract %%%
\begin{abstract}           % abstract environment, this is optional
\par Abstract will be included once all parts are finished
% Those algorithms will be firstly created based on the composite trapezoidal rule. It is the "simplest??"
% or  This thesis investigates how to solve univariate integration problems using numerical methods, including the trapezoidal rule and the Simpson's rule. Most existing guaranteed algorithms are not adaptive and require too much a priori information. Most existing adaptive algorithms do not have valid justification for their results. The goal is to create adaptive algorithms utilizing the two above-mentioned methods with guarantees. The classes of integrands studied in this thesis are cones. The algorithms are analytically proved to be a success if the integrand lies in the cone. The algorithms are adaptive and automatically adjust the computational costs based on the integrand values. The lower and upper bounds on the computational costs for both algorithms are derived. The lower bounds on the complexity of the problems are derived as well. By comparing the upper bounds on the computational cost and the lower bounds on the complexity, our algorithms are shown to be asymptotically optimal. Numerical experiments are implemented.   %you need a separate abstract.tex file to include it.
\end{abstract}


\textpages     % Settings of text-pages are done with \textpages command

% Chapters are created with \Chapter{title} command
\Chapter{INTRODUCTION}

This thesis is motivated by solving univariate integration problems with automatic numerical algorithms. Automatic algorithms conveniently determine the computational effort required to obtain an approximate answer that differs from the true answer by no more than an error tolerance, $\varepsilon$.  The required inputs are both $\varepsilon$ and a black-box routine that provides function values.  Unfortunately, most commonly used adaptive, automatic algorithms are not guaranteed to provide answers satisfying the error tolerance. On the other hand, most existing guaranteed automatic algorithms are not adaptive, i.e., they are do not adjust their effort based on information about the function obtained through sampling.  The goal here is to construct adaptive, automatic algorithms that are guaranteed to satisfy the error tolerance.
% Section are created with \Section{title} command


Automatic algorithms conveniently determine the computational effort required to obtain an approximate answer that differs from the true answer by no more than an error tolerance, $\varepsilon$.  The required inputs are both $\varepsilon$ and a black-box routine that provides function values.  Unfortunately, most commonly used adaptive, automatic algorithms are not guaranteed to provide answers satisfying the error tolerance. On the other hand, most existing guaranteed automatic algorithms are not adaptive, i.e., they are do not adjust their effort based on information about the function obtained through sampling.  The goal here is to construct adaptive, automatic algorithms that are guaranteed to satisfy the error tolerance.

\section{Non-Adaptive, Automatic Algorithms for Balls of Input Functions} \label{nonadaptintrosubsec}
Let $\cf$ be a \emph{linear space} of input functions defined on $\cx$ with \emph{semi-norm} $\Fnorm{\cdot}$, let $\cg$ be a linear space of outputs with \emph{norm} $\norm[\cg]{\cdot}$, and let $S:\cf \to \cg$ be a \emph{solution operator}.  Suppose that one has a sequence of fixed-cost algorithms, $\{A_n\}_{n \in \ci}$, indexed by their computational cost, $n$, with $\ci \subseteq \natzero$.  Furthermore, suppose that there is some known error bound of the form
\begin{subequations} \label{algoerr}
\begin{equation} \label{traditionerra}
\norm[\cg]{S(f)-A_n(f)} \le h(n) \Fnorm{f},
\end{equation}
where $h:\ci \to [0,\infty)$ is non-negative valued and non-increasing. Note that $A_n$ must be exact for input functions with vanishing semi-norms, i.e., $S(f)=A_n(f)$ if $\Fnorm{f}=0$.   Furthermore, $h$ is assumed to have zero infimum, which makes it possible to define $h^{-1}$ for all positive numbers:
\begin{equation} \label{hinvdef}
\inf_{n \in \ci} h(n) = 0, \qquad h^{-1}(\varepsilon) = \min \{n \in \ci : h(n) \le \varepsilon\}, \qquad \varepsilon > 0.
\end{equation}
\end{subequations}
Error bound \eqref{algoerr} allows one to construct an automatic, yet non-adaptive, algorithm that is guaranteed for input functions in a prescribed $\cf$-ball.

\begin{algo}[Non-Adaptive, Automatic] \label{nonadaptalgo} Let $\{A_n\}_{n \in \ci}$ be defined as above, and let $\sigma$ be a fixed positive number.  For any input function $f\in \cb_{\sigma}:=\{ f \in \cf : \Fnorm{f} \le \sigma\}$ and any positive error tolerance $\varepsilon$, find the computational cost needed to satisfy the error tolerance, $n=h^{-1}(\varepsilon/\sigma)$.  Return $A_n(f)$ as the answer.
\end{algo}

\begin{theorem}  \label{NonAdaptDetermThm}  For $\cf$, $\Fnorm{\cdot}$, $\cg$, $\Gnorm{\cdot}$, $S$ as described above, and under the assumptions of Algorithm \ref{nonadaptalgo}, if $f$ lies in the ball $\cb_\sigma$, then the answer provided by Algorithm \ref{nonadaptalgo} must satisfy the error tolerance, i.e., $\norm[\cg]{S(f)-A_n(f)} \le \varepsilon$.
\end{theorem}

Algorithm \ref{nonadaptalgo}, Theorem \ref{NonAdaptDetermThm}, and the other theoretical results in this article related to Algorithm \ref{nonadaptalgo} are essentially known.  They serve as a benchmark to which we may compare our new adaptive algorithms.

Algorithm \ref{nonadaptalgo} has drawbacks.  If it works for $f \in \cf$, it may not work for $cf \in \cf$, where $c>1$, because $cf$ may fall outside the ball $\cb_\sigma$.  Moreover, although error bound \eqref{traditionerra} depends on $\Fnorm{f}$, the computational cost of Algorithm \ref{nonadaptalgo} does not depend on $\Fnorm{f}$.  The cost is the same whether $\Fnorm{f}=\sigma$ or $\Fnorm{f}$ is much smaller than $\sigma$.  This is because Algorithm \ref{nonadaptalgo} is not adaptive.

\section{Adaptive, Automatic Algorithms for Cones of Input Functions} \label{adapintrosec}

Adaptive, automatic algorithms are common in numerical software packages.  Examples include  MATLAB's {\tt quad} and {\tt integral} \cite{MAT8.1}, the quadrature algorithms in the NAG Library \cite{NAG23}, and the MATLAB Chebfun toolbox \cite{TrefEtal12}.  While these adaptive algorithms work well for many cases, they have no rigorous justification. The methods used to determine the computational cost are either heuristics or asymptotic error estimates that do not hold for finite sample sizes.

In this article we derive guaranteed adaptive, automatic algorithms.  These adaptive algorithms use $\{A_n\}_{n \in \ci}$ with known $h$ as described in \eqref{algoerr} and satisfying some additional technical conditions in \eqref{algseqerrcond}.  Rather than assuming an upper bound on $\Fnorm{f}$, our adaptive algorithms use function data to construct \emph{rigorous} upper bounds on $\Fnorm{f}$.  We highlight the requirements here.

The key idea is to identify a suitable semi-norm on $\cf$, $\Ftnorm{\cdot}$, that is weaker than $\Fnorm{\cdot}$, i.e., there exists a positive constant $\tau_{\min}$ for which
\begin{equation} \label{Fspacecondstrong}
\tau_{\min} \Ftnorm{f} \le \Fnorm{f} \qquad \forall f \in \cf.
\end{equation}
Moreover, there must exist a sequence of algorithms, $\{\tF_n\}_{n\in \ci}$, which approximates $\Ftnorm{\cdot}$ and has a two-sided error bound:
\begin{equation} \label{Gerrbds}
-h_{-}(n)\Fnorm{f} \le \Ftnorm{f}- \tF_n(f) \le h_{+}(n) \Fnorm{f}, \qquad \forall f \in \cf,
\end{equation}
for known non-negative valued, non-increasing $h_{\pm}$ satisfying $\inf_{n \in \ci} h_{\pm}(n) = 0$.  The adaptive algorithms to approximate $S$ are defined for a cone of input functions:
\begin{equation} \label{conedef}
\cc_{\tau}=\{f \in \cf : \Fnorm{f} \le \tau \Ftnorm{f} \}.
\end{equation}
(An arbitrary cone is a subset of a vector space that is closed under scalar multiplication.) Although the functions in this cone may have arbitrarily large $\tcf$- and $\cf$-semi-norms, the assumptions above make it
possible to construct reliable, data-driven upper bounds on $\Ftnorm{f}$ and $\Fnorm{f}$.

The above assumptions are all that is required for our two-stage adaptive Algorithm \ref{twostagedetalgo}.  For our multi-stage adaptive Algorithm \ref{multistagealgo}, we further assume that the algorithms $\tF_n$ and $A_n$ use the same function data for all $n \in \ci$.  We also assume that there exists some $r>1$ such that for every $n \in \ci$ there exists an $\tn \in \ci$ satisfying $n < \tn \le rn$ and for which the data for $A_n$ are embedded in the data for $A_{\tn}$. One may think of $r$ as the cost multiple that one might need to incur when moving to the next more costly nested algorithm.

Section \ref{integsec} applies these ideas to the problem of evaluating $\int_0^1 f(x) \, \dif x$.  Here $\cf$ is the set of all continuous functions whose first derivatives have finite (total) variation, $\Fnorm{f}=\Var(f')$, and $\Ftnorm{f}=\norm[1]{f'-f(1)+f(0)}$.  The adaptive algorithm is a composite, equal-width, trapezoidal rule, where the number of trapezoids depends on the data-driven upper bound on $\Var(f')$.  The computational cost is no greater than $4+ \tau + \sqrt{\tau \Var(f')/(4\varepsilon)}$ (Theorem \ref{multistageintegthm}), where $\Var(f')$ is unknown. Here the cone constant $\tau$ is related to the minimum sample size, and $1/\tau$ represents a length scale for possible spikes that one wishes to integrate accurately.


%\Section{The Composite Trapezoidal Rule and Its Error Bound} \label{sec:trap}
%
%A
%
%\Section{The Composite Simpson's Rule and Its Error Bound}
%
%A
%
%\Section{Non-adaptive, automatic algorithms}
%
%A
%
%\Section{Adaptive but flawed algorithms}
%
%A
%
%\Section{Outline}
%
%A


\clearpage



\Chapter{Background}

\Section{Definitions and Assumptions}

\textcolor{red}{I need to write down more details. I need to give all definition before used in the algorithms for both trap and sim.}

The problem to be solved is univariate integration on the interval $[a,b]$, $\INT(f):=\int_{a}^{b}f(t) \, \dif t \in \reals$.  We will The fixed cost building blocks to construct the adaptive integration algorithm are the composite Simpson's rule based on an even number of $3n$ intervals.

The algorithms used in this section on integration is based on linear splines on $[a,b]$.  The node set and the linear spline algorithm using $n$ function values are defined for $n \in \mathcal{I}:=\{2,3,\ldots\}$ as follows:
\begin{subequations} \label{linearspline}
\begin{equation}
x_i=\frac{i-1}{n-1}, \qquad i=1, \ldots, n,
\end{equation}
\begin{multline}
A_{n}(f)(x):=(n-1) \left[ f(x_{i})(x_{i+1}-x) +f(x_{i+1})(x-x_i) \right] \\ \text{for }x_i \leq x \leq x_{i+1}.
\end{multline}
\end{subequations}
The cost of each function value is one and so the cost of  $A_n$ is $n$.




The space of input functions is $\cf:=\mathcal{V}^{1}$, the space of functions whose first derivatives have finite variation.  The general definitions of some relevant norms and spaces are as follows:
\begin{subequations} \label{defSobolev}
\begin{gather}
\Var(f) := \sup_{\substack{n \in \naturals\\ 0 = x_0 < x_1 < \cdots < x_{n} =1}} \sum_{i=1}^n \abs{f(x_i)-f(x_{i-1})}, \\
\norm[p]{f}:= \begin{cases} \displaystyle \left[\int_0^1 \abs{f(x)}^p \, \dif x \right]^{1/p}, & 1 \le p < \infty,\\[1ex]
\displaystyle  \sup_{0 \le x \le 1} \abs{f(x)}, & p=\infty,
\end{cases}
\\
\cv^{k}: =\cv^{k}[0,1]=\{f\in C[0,1]: \Var(f^{(k)}) < \infty \}, \\
\mathcal{W}^{k,p}=\mathcal{W}^{k,p}[0,1]=\{f\in C[0,1]: \|f^{(k)}\|_{p}<\infty\}.
\end{gather}
\end{subequations}
The stronger semi-norm is $\Fnorm{f}:=\Var(f')$, while the weaker semi-norm is
\[
\Ftnorm{f}:=\norm[1]{f'-A_2(f)'}=\norm[1]{f'-f(1)+f(0)}=\Var(f-A_2(f)),
\]
where $A_2(f): x \mapsto f(0)(1-x)+f(1)x$ is the linear interpolant of $f$ using the two endpoints of the integration interval. The reason for defining $\Ftnorm{f}$ this way is that $\Ftnorm{f}$ vanishes if $f$ is a linear function, and linear functions are integrated exactly by the trapezoidal rule.  The cone of integrands is defined as
\begin{equation}\label{coneinteg}
\cc_{\tau}:=\{f\in \cv^{1}:\Var(f')\leq\tau\|f'-f(1)+f(0)\|_1\}.
\end{equation}

The algorithm for approximating $\norm[1]{f'-f(1)+f(0)}$ is the $\tcf$-semi-norm of the linear spline, $A_n(f)$:
\begin{align}
\nonumber
\tF_n(f)&:=\Ftnorm{A_n(f)}=\bignorm[1]{A_n(f)'-A_2(f)'} \\
\label{1direst}
&=\sum_{i=1}^{n-1}\left|f(x_{i+1})-f(x_{i}) - \frac{f(1)-f(0)}{n-1}\right|.
\end{align}
The variation of the first derivative of the linear spline of $f$, i.e.,
\begin{equation} \label{Fnormalg}
F_n(f) :=\Var(A_n(f)') = (n-1)\sum_{i=1}^{n-2} \bigabs{f(x_i) - 2 f(x_{i+1})+f(x_{i+2})},
\end{equation}
provides a lower bound on $\Var(f')$ for $n \ge 3$, and can be used in the necessary condition that $f$ lies in $\cc_\tau$ as described in Remark \ref{neccondrem}.
The Mean Value Theorem implies that
\begin{align*}
F_n(f) &= (n-1)\sum_{i=1}^{n-1} \bigabs{[f(x_{i+2}) - f(x_{i+1})] - [f(x_{i+1}) - f(x_{i})]} \\
&= \sum_{i=1}^{n-1} \abs{f'(\xi_{i+1}) - f'(\xi_{i})} \le \Var(f'),
\end{align*}
where $\xi_i$ is some point in $[x_i,x_{i+1}]$.

\Section{Space for Simpson's Rule}

From (ref), the error bound of Simpson's rule is related to the variation of the third derivatives of the function to be integrated:
\begin{equation}\label{errorboundSimpson}
    \text{err}(f,n)\le\overline{\text{err}}(f,n):=\frac{(b-a)^4\Var(f''')}{5832n^4}.
\end{equation}

We do not have the variation of the third derivative of the function. In order to find the error bound, it is important to find an approximation of variation of the third derivative of the function using only function values.

%$\{t_i\}_{i=0}^{6n}$, where $a=t_{0}\le t_{1}\le\cdots\le t_{6n-1}\le t_{6n}=b$, and partition   and $t_{3j-3}\le x_{j}\le t_{3j}$ for $j=1,\cdots,2n$,
Given any partition $\{x_j\}_{j=0}^{n+1}$, where $a=x_{0}\le x_{1}\le\cdots\le x_{n}\le x_{n+1}=b$, define an approximation to $\Var(f''')$ as:
$$\widehat{V}(f''',\{x_j\}_{j=0}^{n+1})=\sum_{j=1}^{n-1}|f'''(x_{j+1})-f'''(x_{j})|.$$
By definition, the approximation is actually a lower bound of $\Var(f''')$:
\begin{equation}\label{vhatlessvar}
    \widehat{V}(f''',\{x_j\}_{j=0}^{n+1})\leq\Var{(f''')}, \quad \forall f \in \cc, \quad \{x_j\}_{j=0}^{n+1}, \quad n \in \mathbb{N}.
\end{equation}

The algorithm will be guaranteed to work for the cone of integrands for which $\widehat{V}(f''',\{x_j\}_{j=0}^{n+1})$ does not underestimate $\Var{(f''')}$ too much:
%\begin{equation}
\begin{multline}\label{coneinteg}
\cc:=\left\{f\in \cv^{3}, \Var(f''')\leq \mathfrak{C}(\text{size}(\{x_j\}_{j=0}^{n+1}))\widehat{V}(f''',\{x_j\}_{j=0}^{n+1}),\right.\\ \left.\text{for all choices of } n\in \mathbb{N}, \text{and }\{x_j\}_{j=0}^{n+1} \text{with }\text{size}(\{x_j\}_{j=0}^{n+1})<\mathfrak{h}\right\},
\end{multline}
%\end{equation}
where $\mathfrak{C}(\text{size}(\{x_j\}_{j=0}^{n+1}))$ is the inflation factor. The cut-off value $\mathfrak{h}$ and the inflation factor $\mathfrak{C}$ define the cone. The choice of $\mathfrak{C}$ is flexible. But it must be non-decreasing. One choice could be $\mathfrak{C}(h)=\mathfrak{C}(0)\frac{\mathfrak{h}}{\mathfrak{h}-h}, \mathfrak{C}(0)\ge1$.




\Chapter{Error Analysis}
\textcolor{red}{I need to translate trap language to sim language to make them uniform. I need to explain the deduction for trap without using and assumptions or known theories in the paper. I also need to figure out notations. I need the notations not to conflict. Then I need to go to chapter 2 and change notation.}

\Section{Trapezoidal Rule}
Constructing the adaptive algorithm for integration requires an upper bound on the error of $T_n$ and a two-sided bound on the error of $\tF_n$.  Note that $\tF_{n}(f)$ never overestimates $\Ftnorm{f}$ because
\begin{align*}
\Ftnorm{f} & = \bignorm[1]{f'-A_2(f)'}
= \sum_{i=1}^{n-1} \int_{x_i}^{x_{i+1}} \abs{f'(x) - A_2(f)'(x)} \, \dif x \\
& \ge \sum_{i=1}^{n-1} \abs{\int_{x_i}^{x_{i+1}} [f'(x) - A_2(f)'(x)] \, \dif x}=\norm[1]{A_n(f)'-A_2(f)'} = \tF_n(f).
\end{align*}
Thus, $h_{-}(n):=0$ and $\fc_n=\tfc_n=1$.

To find an upper bound on $\Ftnorm{f}-\tF_{n}(f)$, note that
\begin{equation*}
\Ftnorm{f} - \tF_{n}(f) = \Ftnorm{f} - \bigabs{A_n(f)}_{\tcf} \le \bigabs{f-A_n(f)}_{\tcf} = \bignorm[1]{f' -A_n(f)'},
\end{equation*}
since $(f-A_n(f))(x)$ vanishes for $x=0,1$.  Moreover,
\begin{equation} \label{onenormfp}
\bignorm[1]{f' -A_n(f)'} = \sum_{i=1}^{n-1} \int_{x_i}^{x_{i+1}} \abs{f'(x) -(n-1)[f(x_{i+1})-f(x_i)]} \, \dif x.
\end{equation}

\Subsection{Upperbound}

Constructing the adaptive algorithm for integration requires an upper bound on the error of $T_n$ and a two-sided bound on the error of $\tF_n$.  Note that $\tF_{n}(f)$ never overestimates $\Ftnorm{f}$ because
\begin{align*}
\Ftnorm{f} & = \bignorm[1]{f'-A_2(f)'}
= \sum_{i=1}^{n-1} \int_{x_i}^{x_{i+1}} \abs{f'(x) - A_2(f)'(x)} \, \dif x \\
& \ge \sum_{i=1}^{n-1} \abs{\int_{x_i}^{x_{i+1}} [f'(x) - A_2(f)'(x)] \, \dif x}=\norm[1]{A_n(f)'-A_2(f)'} = \tF_n(f).
\end{align*}
Thus, $h_{-}(n):=0$ and $\fc_n=\tfc_n=1$.

To find an upper bound on $\Ftnorm{f}-\tF_{n}(f)$, note that
\begin{equation*}
\Ftnorm{f} - \tF_{n}(f) = \Ftnorm{f} - \bigabs{A_n(f)}_{\tcf} \le \bigabs{f-A_n(f)}_{\tcf} = \bignorm[1]{f' -A_n(f)'},
\end{equation*}
since $(f-A_n(f))(x)$ vanishes for $x=0,1$.  Moreover,
\begin{equation} \label{onenormfp}
\bignorm[1]{f' -A_n(f)'} = \sum_{i=1}^{n-1} \int_{x_i}^{x_{i+1}} \abs{f'(x) -(n-1)[f(x_{i+1})-f(x_i)]} \, \dif x.
\end{equation}
Now we bound each integral in the summation.  For $i=1, \ldots, n-1$, let $\eta_i(x) = f'(x) -(n-1)[f(x_{i+1})-f(x_i)]$, and let $p_i$ denote the probability that $\eta_i(x)$ is non-negative:
\[
p_i = (n-1)\int_{x_i}^{x_{i+1}} \bbone_{[0,\infty)} (\eta_i(x)) \, \dif x,
\]
and so $1-p_i$ is the probability that $\eta_i(x)$ is negative.  Since $\int_{x_i}^{x_{i+1}} \eta_i(x) \, \dif x =0$, we know that $\eta_i$ must take on both non-positive and non-negative values.  Invoking the Mean Value Theorem, it follows that
\begin{multline*}
\frac{p_i}{n-1} \sup_{x_i \le x \le x_{i+1}} \eta_i(x) \ge \int_{x_i}^{x_{i+1}} \max(\eta_i(x),0) \, \dif x \\
= \int_{x_i}^{x_{i+1}} \max(-\eta_i(x),0) \, \dif x \le \frac{-(1-p_i)}{n-1} \inf_{x_i \le x \le x_{i+1}} \eta_i(x) .
\end{multline*}
These bounds allow us to derive bounds on the integrals in \eqref{onenormfp}:
\begin{align*}
\MoveEqLeft{\int_{x_i}^{x_{i+1}} \abs{\eta_i(x)} \, \dif x} \\
 &= \int_{x_i}^{x_{i+1}} \max(\eta_i(x),0) \, \dif x + \int_{x_i}^{x_{i+1}} \max(-\eta_i(x),0) \, \dif x\\
&=2(1-p_i) \int_{x_i}^{x_{i+1}} \max(\eta_i(x),0) \, \dif x + 2p_i\int_{x_i}^{x_{i+1}} \max(-\eta_i(x),0) \, \dif x\\
&\le \frac{2p_i(1-p_i)}{n-1} \left[ \sup_{x_i \le x \le x_{i+1}} \eta_i(x) - \inf_{x_i \le x \le x_{i+1}} \eta_i(x) \right]\\
&\le\frac{1}{2(n-1)} \left[ \sup_{x_i \le x \le x_{i+1}} f'(x) - \inf_{x_i \le x \le x_{i+1}} f'(x) \right],
\end{align*}
since $p_i(1-p_i)\le 1/4$.

Plugging this bound into \eqref{onenormfp} yields
\begin{align*}
\bignorm[1]{f'-f(1)+f(0)} - \tF_n(f) &= \Ftnorm{f} - \tF_{n}(f)\\
 & \le \bignorm[1]{f' -A_n(f)'}\\
&\le \frac{1}{2n-2} \sum_{i=1}^{n-1} \left[ \sup_{x_i \le x \le x_{i+1}} f'(x) - \inf_{x_i \le x \le x_{i+1}} f'(x) \right] \\
& \le \frac{\Var(f')}{2n-2} = \frac{\Fnorm{f}}{2n-2},
\end{align*}
and so
\begin{equation*}\label{factor}
h_{+}(n):= \frac{1}{2n-2}, \qquad \mathfrak{C}_n =\frac{1}{1 - \tau/(2n-2)} \qquad \text{for } n>1+\tau/2.
\end{equation*}
Since $\tF_2(f)=0$ by definition, the above inequality for $\Ftnorm{f} - \tF_{2}(f)$ implies that
\begin{equation*} \label{taumininteg}
2\bignorm[1]{f'-f(1)+f(0)} = 2 \Ftnorm{f} \le \Fnorm{f} = \Var(f'), \qquad \tau_{\min}=2.
\end{equation*}

The error of the trapezoidal rule in terms of the variation of the first derivative of the integrand is given in \cite[(7.15)]{BraPet11a}:
%\begin{subequations} \label{integhhtilde}
\begin{gather*}
\abs{\int_0^1 f(x) \, dx - T_n(f)} \le h(n) \Var(f') \\
h(n):= \frac{1}{8(n-1)^2}, \qquad h^{-1}(\varepsilon) = \left \lceil \sqrt{\frac{1}{8\varepsilon}} \right \rceil +1.
\end{gather*}
%\end{subequations}
Given the above definitions of $h, \fC_n, \fc_n$, and $\tfc_n$, it is now possible to also specify
\begin{subequations} \label{simplifycond}
\begin{gather}
h_1(n) = h_2(n) = \fC_n h(n) = \frac{1}{4(n-1)(2n-2-\tau)}, \\
h_1^{-1}(\varepsilon) = h_2^{-1}(\varepsilon) = 1+ \left \lceil \sqrt{\frac{\tau}{8 \varepsilon} + \frac{\tau^2}{16}} +\frac{\tau}{4} \right \rceil \le 2 + \frac{\tau}{2} + \sqrt{\frac{\tau}{8\varepsilon}}.
\end{gather}
Moreover, the left side of \eqref{multistageconv}, the stopping criterion inequality in the multi-stage algorithm, becomes
\begin{equation}
\tau h(n_i)\fC_{n_i} \tF_{n_i}(f) = \frac{\tau  \tF_{n_i}(f) } {4(n_i-1)(2n_i-2 -\tau)}.
\end{equation}
\end{subequations}

\begin{theorem} \label{multistageintegthm}
Let $\sigma >0$ be some fixed parameter, and let $\cb_{\sigma}=\{f \in  \mathcal{V}^{1} : \Var(f')\leq \sigma\}$. Let $T \in \ca(\cb_{\sigma}, \reals, \INT, \Lambda^{\std})$ be the non-adaptive trapezoidal rule defined by Algorithm \ref{nonadaptalgo}, and let $\varepsilon>0$ be the error tolerance. Then this algorithm succeeds for $f \in \cb_{\sigma}$, i.e., $\abs{\INT(f) - T(f,\varepsilon)} \le \varepsilon$, and the cost of this algorithm is $\left \lceil \sqrt{\sigma/(8\varepsilon)}\right \rceil + 1$, regardless of the size of $\Var(f')$.

Now let $T \in \ca(\cc_{\tau}, \reals, \INT, \Lambda^{\std})$ be the adaptive trapezoidal rule defined by Algorithm \ref{multistageintegalgo}, and let $\tau$, $n_1$, and $\varepsilon$ be as described there. Let $\cc_\tau$ be the cone of functions defined in \eqref{coneinteg}.  Then it follows that Algorithm \ref{multistageintegalgo} is successful for all functions in $\cc_{\tau}$,  i.e.,  $\abs{\INT(f) - T(f,\varepsilon)} \le \varepsilon$.  Moreover, the cost of this algorithm is bounded below and above as follows:
\begin{multline}
\max \left(\left \lceil\frac{\tau+1}{2} \right \rceil, \left \lceil \sqrt{\frac{ \Var(f')}{8\varepsilon}} \right \rceil \right) +1 \\
\le \max \left(\left \lceil\frac{\tau+1}{2} \right \rceil, \left \lceil \sqrt{\frac{\tau \norm[1]{f'-f(1)+f(0)}}{8\varepsilon}} \right \rceil \right) +1 \\
\le
\cost(T,f;\varepsilon) \\
\le \sqrt{\frac{\tau \norm[1]{f'-f(1)+f(0)}}{2\varepsilon}} + \tau + 4
\le \sqrt{\frac{\tau \Var(f') }{4\varepsilon}} + \tau + 4.
\end{multline}
The algorithm is computationally stable, meaning that the minimum and maximum costs for all integrands, $f$, with fixed $\norm[1]{f'-f(1)+f(0)}$ or $\Var(f')$ are an $\varepsilon$-independent constant of each other.
\end{theorem}

\Subsection{Lowerbound}

Next, we derive a lower bound on the cost of approximating functions in the ball $\cb_{\sigma}$ and in the cone $\cc_{\tau}$ by constructing fooling functions. Following the arguments of Section \ref{LowBoundSec}, we choose  the triangle shaped function $f_0: x \mapsto 1/2-\abs{1/2-x}$. Then
\begin{gather*}
\Ftnorm{f_0}=\norm[1]{f'_0-f_0(1)+f_0(0)}=\int_0^1 \abs{\sign(1/2-x)} \, \dif x = 1, \\ \Fnorm{f_0}=\Var(f'_0)=2= \tau_{\min}.
\end{gather*}
For any $n \in \cj:=\natzero$, suppose that the one has the data $L_i(f)=f(\xi_i)$, $i=1, \ldots, n$ for arbitrary $\xi_i$, where $0=\xi_0 \le \xi_1 < \cdots < \xi_n \le \xi_{n+1} = 1$.  There must be some $j=0, \ldots, n$ such that $\xi_{j+1} - \xi_j \ge 1/(n+1)$.  The function $f_{1}$ is defined as a triangle function on the interval $[\xi_j, \xi_{j+1}]$:
$$
f_{1}(x):=\begin{cases} \displaystyle
\frac{\xi_{j+1}-\xi_{j}-\abs{\xi_{j+1}+\xi_{j}-2x}}{8} & \xi_{j} \le x \leq \xi_{j+1},\\
0 & \text{otherwise}.
\end{cases}
$$
This is a piecewise linear function whose derivative changes from $0$ to $1/4$ to $-1/4$ to $0$ provided $0 < \xi_j < \xi_{j+1} < 1$, and so $\Fnorm{f_1}=\Var(f'_1)\le 1$. Moreover,
\begin{gather*}
\INT(f)=\int_0^1 f_1(x) \, \dif x = \frac{(\xi_{j+1} - \xi_j)^2}{16} \ge \frac{1}{16(n+1)^2} =: g(n),\\
g^{-1}(\varepsilon)=\left \lceil \sqrt{\frac{1}{16 \varepsilon}} \right \rceil - 1.
\end{gather*}
Using these choices of $f_0$ and $f_1$, along with the corresponding $g$ above, one may invoke Theorems \ref{complowbdball}--\ref{complowbd}, and Corollary \ref{optimcor} to obtain the following theorem.

\begin{theorem} \label{complowbdinteg} For $\sigma>0$ let $\cb_{\sigma}=\{f \in \cv^{1} : \Var(f') \le \sigma\}$.  The complexity of integration on this ball is bounded below as
\begin{equation*}
\comp(\varepsilon,\ca(\cb_{\sigma},\reals,\INT,\Lambda^{\std}),\cb_{s}) \ge \left \lceil \sqrt{\frac{\min(s,\sigma)}{16 \varepsilon}} \right \rceil -1 .
\end{equation*}
Algorithm \ref{nonadaptalgo} using the trapezoidal rule has optimal order in the sense of Theorem \ref{optimalprop}.

For $\tau>2$, the complexity of the integration problem over the cone of functions $\cc_{\tau}$ defined in \eqref{coneinteg} is bounded below as
\begin{equation*}
\comp(\varepsilon,\ca(\cc_{\tau},\reals,\INT,\Lambda^{\std}),\cb_{s}) \ge \left \lceil \sqrt{\frac{(\tau-2)s}{32 \tau \varepsilon}} \right \rceil -1 .
\end{equation*}
The adaptive trapezoidal Algorithm \ref{multistageintegalgo} has optimal order for integration of functions in $\cc_{\tau}$ in the sense of Corollary \ref{optimcor}.
\end{theorem}

\Section{Simpson's Rule}

From (ref), the error bound of Simpson's rule is related to the variation of the third derivatives of the function to be integrated:
\begin{equation}\label{errorboundSimpson}
    \text{err}(f,n)\le\overline{\text{err}}(f,n):=\frac{(b-a)^4\Var(f''')}{5832n^4}.
\end{equation}

We do not have the variation of the third derivative of the function. In order to find the error bound, it is important to find an approximation of variation of the third derivative of the function using only function values.

%$\{t_i\}_{i=0}^{6n}$, where $a=t_{0}\le t_{1}\le\cdots\le t_{6n-1}\le t_{6n}=b$, and partition   and $t_{3j-3}\le x_{j}\le t_{3j}$ for $j=1,\cdots,2n$,
Given any partition $\{x_j\}_{j=0}^{n+1}$, where $a=x_{0}\le x_{1}\le\cdots\le x_{n}\le x_{n+1}=b$, define an approximation to $\Var(f''')$ as:
$$\widehat{V}(f''',\{x_j\}_{j=0}^{n+1})=\sum_{j=1}^{n-1}|f'''(x_{j+1})-f'''(x_{j})|.$$
By definition, the approximation is actually a lower bound of $\Var(f''')$:
\begin{equation}\label{vhatlessvar}
    \widehat{V}(f''',\{x_j\}_{j=0}^{n+1})\leq\Var{(f''')}, \quad \forall f \in \cc, \quad \{x_j\}_{j=0}^{n+1}, \quad n \in \mathbb{N}.
\end{equation}

The algorithm will be guaranteed to work for the cone of integrands for which $\widehat{V}(f''',\{x_j\}_{j=0}^{n+1})$ does not underestimate $\Var{(f''')}$ too much:
%\begin{equation}
\begin{multline}\label{coneinteg}
\cc:=\left\{f\in \cv^{3}, \Var(f''')\leq \mathfrak{C}(\text{size}(\{x_j\}_{j=0}^{n+1}))\widehat{V}(f''',\{x_j\}_{j=0}^{n+1}),\right.\\ \left.\text{for all choices of } n\in \mathbb{N}, \text{and }\{x_j\}_{j=0}^{n+1} \text{with }\text{size}(\{x_j\}_{j=0}^{n+1})<\mathfrak{h}\right\},
\end{multline}
%\end{equation}
where $\mathfrak{C}(\text{size}(\{x_j\}_{j=0}^{n+1}))$ is the inflation factor. The cut-off value $\mathfrak{h}$ and the inflation factor $\mathfrak{C}$ define the cone. The choice of $\mathfrak{C}$ is flexible. But it must be non-decreasing. One choice could be $\mathfrak{C}(h)=\mathfrak{C}(0)\frac{\mathfrak{h}}{\mathfrak{h}-h}, \mathfrak{C}(0)\ge1$.

%Approximation of $\Var(f''')$ using only function values:
We cannot use $\widehat{V}(f''',\{x_j\}_{j=0}^{n+1})$ to approximate $\Var{(f''')}$ because it depends on values of $f'''$, not values of $f$. However, $\widehat{V}(f''',\{x_j\}_{j=0}^{n+1})$ is closely related to the following approximation to $\Var{(f''')}$:
\begin{multline}\label{vtilde}
\widetilde{V}_n(f)=\frac{27n^3}{(b-a)^3}\sum_{j=1}^{n-1}\left|f(t_{3j+3})-3f(t_{3j+2})+3f(t_{3j+1})\right.\\\left.-2f(t_{3j})+3f(t_{3j-1})-3f(t_{3j-2})+f(t_{3j-3})\right|,
\end{multline}
where the $t_{i}$'s are uniformly distributed between $[a,b]$

\begin{equation}\label{datapoints}
t_i=a+\frac{i(b-a)}{3n}, \qquad i=0, \ldots, 3n, \qquad n\in\mathbb{N}.
\end{equation}

% may be say more
% Since $$\frac{216n^3}{(b-a)^3}\left|f(t_{i-3})-3f(t_{i-2})+3f(t_{i-1})-f(t_{i})\right|=f'''(x_{i-1}),???$$ for some $x_{i-1} \in [t_{3i-3},t_{3i}]$,
We use divided differences to explain \eqref{vtilde}. Let $h=t_{i+1}-t_{i}=(b-a)/3n$ and
\begin{align*}
  f[t_{i}]&=f(t_{i}), \text{ for } i=0,\cdots, 3n,\\
  f[t_{i},t_{i-1}]&=\frac{f(t_{i})-f(t_{i-1})}{h},\text{ for } i=1, \cdots, 3n,\\
  f[t_{i},t_{i-1},t_{i-2}]&=\frac{f(t_{i})-2f(t_{i-1})+f(t_{i-2})}{2h^2},\text{ for } i=2, \cdots, 3n,\\
  f[t_{i},t_{i-1},t_{i-2},t_{i-3}]&=\frac{f(t_{i})-3f(t_{i-1})+3f(t_{i-2})-f(t_{i-3})}{6h^3}, \text{ for } i=3, \cdots, 3n.
\end{align*}

According to Mean Value Theorem for divided differences, (ref), for all $j=1,2,\cdots,n$, $\exists x_j\in (t_{3j-3},t_{3j})$ such that
\begin{equation*}
    f[t_{3j},t_{3j-1},t_{3j-2},t_{3j-3}]=\frac{f'''(x_j)}{6},
\end{equation*}
for $j = 1, 2, \cdots, n.$ This implies that
\begin{multline}\label{vtileqftriprime}
  f'''(x_j)=\frac{f(t_{3j})-3f(t_{3j-1})+3f(t_{3j-2})-f(t_{3j-3})}{h^3},\\=\frac{27n^3}{(b-a)^3}[f(t_{3j})-3f(t_{3j-1})+3f(t_{3j-2})-f(t_{3j-3})].
\end{multline}



If we combine \eqref{vtilde} and \eqref{vtileqftriprime} together, we obtain
\begin{equation}\label{vtileqvhat}
    \widetilde{V}_n(f)=\sum_{j=1}^{n-1}\left|f'''(x_{j+1})-f'''(x_{j})\right|=\widehat{V}(f''',\{x_j\}_{j=0}^{n+1}).
\end{equation}
Then we can use $\widetilde{V}_n(f)$ to approximate $\Var(f''')$ by just using function values.

\Subsection{Upperbound}

\begin{theorem}\label{uppbndcost}
    Let $N(f,\varepsilon)$ denote the final number of $n_l$ in Stage 2 when the algorithm terminates. Then this number is bounded below and above in terms of the true, yet unknown, $\Var(f''')$.
    \begin{multline}\label{uppbndcostineq}
        \max\left(\left\lfloor\frac{2(b-a)}{\mathfrak{h}}\right\rfloor+1,\left\lceil(b-a)\left(\frac{\Var(f''')}{5832\varepsilon}\right)^{1/4}\right\rceil\right)\leq N(f,\varepsilon)\\ \leq 2\min\left\{n\in\mathbb{N}:n\geq2\left(\left\lfloor\frac{(b-a)}{\mathfrak{h}}\right\rfloor+1\right),\eta(n)\Var(f''')\leq\varepsilon\right\}\\ \leq 2\min_{0<\alpha\leq1}\max\left(2\left(\left\lfloor\frac{(b-a)}{\alpha\mathfrak{h}}\right\rfloor+1\right),(b-a)\left(\frac{\mathfrak{C}(\alpha\mathfrak{h})\Var(f''')}{5832\varepsilon}\right)^{1/4}+1\right).
    \end{multline}
    The number of function values required by the algorithm is $3N(f,\varepsilon)+1$.
\end{theorem}
\begin{proof}
  %No matter what inputs $f$ and $\varepsilon$ are provided, the number of intervals must be at least $n_1=\lfloor2(b-a)/\mathfrak{h}\rfloor+1$ in order to comply with both Simpson's rule and divided differences method. Then the number of intervals increases until $\widetilde{\text{err}}(f,n)\le\varepsilon$, which by \eqref{errorboundcone} implies that $\overline{\text{err}}(f,n)\le\varepsilon$. This implies the lower bound on $N(f,\varepsilon)$.
  No matter what inputs $f$ and $\varepsilon$ are provided, $N(f,\varepsilon)\ge n_1=2(\lfloor (b-a)/\mathfrak{h}\rfloor+1)$. Then the number of intervals increases until $\widetilde{\text{err}}(f,n)\le\varepsilon$, which by \eqref{errorboundcone} implies that $\overline{\text{err}}(f,n)\le\varepsilon$. This implies the lower bound on $N(f,\varepsilon)$.

  Let $L$ be the value of $l$ for which Algorithm \ref{multistageintegalgosimpson} terminates. Since $n_1$ satisfies the upper bound, we may assume that $L \ge 2$. Let $m$ be the integer found in Step 3, and let $m^*=\max(2,m)$. Note that $\eta((m^*-1)n_{L-1})\Var(f''')>\varepsilon$. For $m^*=2$, this is true because $\eta(n_{L-1})\Var(f''')\ge\eta(n_{L-1})\widetilde{V}_{n_{L-1}}(f)=\widetilde{\text{err}}(f,n_{L-1})>\varepsilon$. For $m^*=m>2$ it is true because of the definition of $m$. Since $\eta$ is a decreasing function, it follows that
  $$(m^*-1)n_{L-1}<n^*:=\min\left\{n\in\mathbb{N}:n\ge\left\lfloor\frac{2(b-a)}{n}\right\rfloor+1,\eta(n)\Var(f''')\le\varepsilon\right\}.$$
  Therefore $n_L=m^*n_{L-1}<m^*\frac{n^*}{m^*-1}=\frac{m^*}{m^*-1}n^*\le2n^*$.

  To prove the latter part of the upper bound, we need to prove that
  $$n^*\leq\max\left(\left\lfloor\frac{2(b-a)}{\alpha\mathfrak{h}}\right\rfloor+1,(b-a)\left(\frac{\mathfrak{C}(\alpha\mathfrak{h})\Var(f''')}{5832\varepsilon}\right)^{1/4}+1\right),\quad 0<\alpha<1.$$
  For fixed $\alpha\in(0,1]$, we only need to consider that case where $n^*>\left\lfloor2(b-a)/(\alpha\mathfrak{h})\right\rfloor+1$. This implies that $n^*-1>\left\lfloor2(b-a)/(\alpha\mathfrak{h})\right\rfloor\ge 2(b-a)/(\alpha\mathfrak{h})$ thus $\alpha\mathfrak{h}\ge2(b-a)/(n^*-1)$. Also by the definition of $n^*$, $\eta$, and $\mathfrak{C}$ is non-decreasing:
  \begin{align*}
    &\eta(n^*-1)\Var(f''')>\varepsilon, \\
    \Rightarrow 1&<\left(\frac{\eta(n^*-1)\Var(f''')}{\varepsilon}\right)^{1/4},\\
    \Rightarrow n^*-1&<n^*-1\left(\frac{\eta(n^*-1)\Var(f''')}{\varepsilon}\right)^{1/4},\\
    &=n^*-1\left(\frac{(b-a)^4\mathfrak{C}(2(b-a)/(n^*-1))\Var(f''')}{5832(n^*-1)^4\varepsilon}\right)^{1/4},\\
    &\le(b-a)\left(\frac{\mathfrak{C}(\alpha\mathfrak{h})\Var(f''')}{5832\varepsilon}\right)^{1/4}.
  \end{align*}
  This completes the prove of latter part of the upper bound.
\end{proof}


\Subsection{Lowerbound}

building fooling function:
\begin{subequations} \label{bumpfunction}
%\begin{gather}
%bump(x;t,h):= \begin{cases} \displaystyle (x-t)^3/6, & t \le x < t+h,\\[1ex]
%\displaystyle [(x-t)^2(t+2h-x)+(x-t)(t+3h-x)(x-t-h)+(t+4h-x)(x-t-h)^2]/6, & t+h \le x < t+2h,\\[1ex]
%\displaystyle [(x-t)(t+3h-x)^2+(t+4h-x)(x-t-h)(t+3h-x)+(t+4h-x)^2(x-t-2h)]/6, & t+2h \le x < t+3h,\\[1ex]
%\displaystyle (t+4h-x)^3/6, & t+3h \le x < t+4h,\\[1ex]
%\displaystyle  0, & \text{otherwise},
%\end{cases}
%\\
\begin{gather}
\text{bump}(x;t,h):= \begin{cases} \displaystyle (x-t)^3/6, & t \le x < t+h,\\[1ex]
\displaystyle [-3(x-t)^3+12h(x-t)^2-12h^2(x-t)+4h^3]/6, & t+h \le x < t+2h,\\[1ex]
\displaystyle [3(x-t)^3-24h(x-t)^2+60h^2(x-t)-44h^3]/6, & t+2h \le x < t+3h,\\[1ex]
\displaystyle (t+4h-x)^3/6, & t+3h \le x < t+4h,\\[1ex]
\displaystyle  0, & \text{otherwise},
\end{cases}
\\
\text{bump}'''(x;t,h):= \begin{cases} \displaystyle 1, & t \le x < t+h,\\[1ex]
\displaystyle -3, & t+h \le x < t+2h,\\[1ex]
\displaystyle 3, & t+2h \le x < t+3h,\\[1ex]
\displaystyle -1, & t+3h \le x < t+4h,\\[1ex]
\displaystyle  0, & \text{otherwise},
\end{cases}, \\
\Var(\text{bump}'''(\cdot;t,h))\le 16 \text{ with equality if } a<t<t+4h<b, \\
\int_{a}^{b}\text{peak}(x;t,h)dx=h^4.
\end{gather}
\end{subequations}

The following double-bump function always lies in $\cc$:
\begin{subequations}
    \begin{multline}\label{foolingfunction}
        \text{twobp}(x;t,h,\pm):=\text{bump}(x;a,\mathfrak{h})\pm\frac{15[\mathfrak{C}(h)-1]}{16}\text{bump}(x;t,h)\\ a+5\mathfrak{h}\le h \le b-5h, 0\le h <\mathfrak{h}.
    \end{multline}
    \\
    \begin{equation}
        \Var(\text{twobp}'''(x;t,h,\pm))=15+16\frac{15[\mathfrak{C}(h)-1]}{16}=15\mathfrak{C}(h).
    \end{equation}
\end{subequations}
From this definition it follows that
\begin{align*}
  &\mathfrak{C}(\text{size}(\{x_j\}_{j=0}^{n+1}))\widehat{V}(\text{twobp}'''(x;t,h,\pm),\{x_j\}_{j=0}^{n+1})\\
  \ge & \begin{cases} \displaystyle 15\mathfrak{C}(h)=\Var(\text{twobp}'''(x;t,h,\pm)), h \le \text{size}(\{x_j\}_{j=0}^{n+1})) <\mathfrak{h}\\[1ex]
                      \displaystyle \mathfrak{C}(0)\Var(\text{twobp}'''(x;t,h,\pm)), 0\le \text{size}(\{x_j\}_{j=0}^{n+1}))<h
        \end{cases}\\
  \ge & \Var(\text{twobp}'''(x;t,h,\pm))
\end{align*}

Although $\text{twobp}'''(x;t,h,\pm)$ may have a bump with arbitrarily small width $4h$, the height is small enough for $\text{twobp}'''(x;t,h,\pm)$ to lie in the cone.


complexity:

\begin{theorem}\label{lowbndcost}
    Let $int$ be any (possibly adaptive) algorithm that succeeds for all integrands in $\cc$, and only uses function values. For any error tolerance $\varepsilon > 0$ and any arbitrary value of $\Var(f''')$, there will be some $f\in \cc$ for which $int$ must use at least
    \begin{equation}\label{lowbndcostineq}
        -\frac{5}{4}+\frac{b-a-5\mathfrak{h}}{8}\left[\frac{[\mathfrak{C}(0)-1]\Var( f''')}{\varepsilon}\right]^{1/4}
    \end{equation}
    function values. As $\Var(f''')/\varepsilon \rightarrow \infty$ the asymptotic rate of increase is the same as the computational cost of \texttt{integral}.
\end{theorem}
\begin{proof}
  For any positive $\alpha$, suppose that $\texttt{int}(\cdot,a,b,\varepsilon)$ evaluates integrand $\alpha\text{bump}'''(\cdot;t,h)$ at $n$ nodes before returning to an answer. Let $\{x_j\}_{j=1}^{m})$ be the $m<n$ ordered nodes used by $\texttt{int}(\cdot,a,b,\varepsilon)$ that fall in the interval $(x_{0},x_{m+1})$ where $x_{0}:=a+3\mathfrak{h}$, $x_{m+1}:=b-h$ (why $h$ but not $\mathfrak{h}$ or $5h$?) and $h:=(b-a-5\mathfrak{h})/(4n+5)$. There must e at least one of these $x_{j}$ with $i=0,\cdots,m$ for which
  \begin{align*}
    \frac{x_{j+1}-x_{j}}{4}\ge\frac{x_{m+1}-x_{0}}{4(m+1)}\ge\frac{x_{m+1}-x_{0}}{4(n+1)}=\frac{b-a-5\mathfrak{h}-h}{4n+4}=h.
  \end{align*}
  Choose one such $x_{j}$ and call it $t$. The choice of $t$ and $h$ ensures that $\texttt{int}(\cdot,a,b,\varepsilon)$ cannot distinguish between $\alpha\text{bump}(\cdot;t,h)$ and $\alpha\text{twobp}(\cdot;t,h,\pm)$. Thus
  \begin{align*}
    \texttt{int}(\alpha\text{twobp}(\cdot;t,h,\pm),a,b,\varepsilon)=\texttt{int}(\alpha\text{bump}(\cdot;t,h),a,b,\varepsilon)
  \end{align*}
  Moreover, $\alpha\text{bump}(\cdot;t,h)$ and $\alpha\text{twobp}(\cdot;t,h,\pm)$ are all in the cone $\cc$. This means that $\texttt{int}$ is successful for all of the functions.
  \begin{subequations}
  \begin{multline*}
    \varepsilon\ge\frac{1}{2}\left[\right.\left|\int_{a}^{b}\alpha\text{twobp}(x;t,h,-)dx-\texttt{int}(\alpha\text{twobp}(\cdot;t,h,-),a,b,\varepsilon)\right|\\
    +\left|\int_{a}^{b}\alpha\text{twobp}(x;t,h,+)dx-\texttt{int}(\alpha\text{twobp}(\cdot;t,h,+),a,b,\varepsilon)\right|\left.\right]
  \end{multline*}
  \begin{multline*}
    \ge\frac{1}{2}\left[\right.\left|\texttt{int}(\alpha\text{bump}(\cdot;t,h,-),a,b,\varepsilon)-\int_{a}^{b}\alpha\text{twobp}(x;t,h,-)dx\right|\\
    +\left|\int_{a}^{b}\alpha\text{twobp}(x;t,h,+)dx-\texttt{int}(\alpha\text{bump}(\cdot;t,h,+),a,b,\varepsilon)\right|\left.\right]
  \end{multline*}
  \begin{align*}
     &\ge\frac{1}{2}\left|\int_{a}^{b}\alpha\text{twobp}(x;t,h,+)dx-\int_{a}^{b}\alpha\text{twobp}(x;t,h,-)dx\right|\\
     &=\int_{a}^{b}\alpha\texttt{bump}(x;t,h)dx\\
     &=\frac{15\alpha[\mathfrak{C}(h)-1]h^4}{16}\\
     &=\frac{[\mathfrak{C}(h)-1]h^4\Var(\alpha\texttt{bump}'''(\cdot;a,\mathfrak{h}))}{16}
  \end{align*}
  \end{subequations}
  Substituting $h$  in terms of $n$:
      \begin{align*}
        4n+5=\frac{b-a-5\mathfrak{h}}{h}&\ge(b-a-5\mathfrak{h})\left[\frac{[\mathfrak{C}(h)-1]\Var(\alpha \texttt{bump}'''(\cdot;a,\mathfrak{h})))}{16\varepsilon}\right]^{1/4},\\
        &\ge\frac{b-a-5\mathfrak{h}}{2}\left[\frac{[\mathfrak{C}(0)-1]\Var(\alpha \texttt{bump}'''(\cdot;a,\mathfrak{h}))}{\varepsilon}\right]^{1/4}.
    \end{align*}
    Since $\alpha$ is an arbitrary positive number, the value of $\Var(\alpha \texttt{bump}'''(\cdot;a,\mathfrak{h}))$ is arbitrary.

    Finally, comparing the upper bound on the computational cost of $\texttt{integral}$ in \eqref{uppbndcostineq} with the lower bound on the computational cost of the best algorithm in \eqref{lowbndcostineq}, both of them increase as $\mathcal{O}((\Var(f''')/\varepsilon))^{1/4}$ as $(\Var(f''')/\varepsilon)^{1/4}\rightarrow \infty$. Thus $\texttt{integral}$ is optimal.
\end{proof}

\Chapter{Adaptive, Automatic Algorithms with Guarantees}

\Section{Basic Concepts}

\textcolor{red}{I need to explain the embedded mechanism and stopping criteria with tolerance, max iteration and max number of points. In the following subsections, I need detailed explanation of the algorithms. I also need to find out where to put the algorithms because how I got the guarantees (error less than tolerance) is in Chapter 4.}

\Section{Trapezoidal Rule}


\begin{algo}[Adaptive Univariate Integration] \label{multistageintegalgo}
Let the sequence of algorithms $\{T_n\}_{n\in \mathcal{I}}$, $\{\tF_n\}_{n\in \mathcal{I}}$, and $\{F_n\}_{n\in \mathcal{I}}$ be as described above.
Let $\tau\ge2$ be the cone constant. Set $i=1$. Let $n_1=\lceil(\tau+1)/2\rceil+1$. For any error tolerance $\varepsilon$ and input function $f$, do the following:
\begin{description}
\item[Stage 1.\ Estimate {$\norm[1]{f'-f(1)+f(0)}$} and bound {$\Var(f')$}.] Compute $\tF_{n_i}(f)$ in \eqref{1direst} and $F_{n_i}(f)$ in \eqref{Fnormalg}.

\item[Stage 2. Check the necessary condition for $f \in \cc_{\tau}$.] Compute
    \begin{align*}
     \tau_{\min,n_i} =  \frac{F_{n_i}(f)}{\tF_{n_i}(f)+F_{n_i}(f)/(2n_i-2)}.
    \end{align*}
If $\tau \ge \tau_{\min,n_i}$, then go to stage 3.  Otherwise, set $\tau = 2\tau_{\min,n_i}$.  If $n_i \ge (\tau+1)/2$, then go to stage 3.  Otherwise, choose
$$
n_{i+1}=1+ (n_i-1)\left\lceil\frac{\tau+1}{2n_i-2}\right\rceil.
$$
Go to Stage 1.

\item[Stage 3. Check for convergence.] Check whether $n_i$ is large enough to satisfy the error tolerance, i.e.
    \begin{equation*}
     \tF_{n_i}(f) \le \frac{4\varepsilon(n_i-1)(2n_i-2 - \tau)}{\tau}.
    \end{equation*}
If this is true, then return $T_{n_i}(f)$ and terminate the algorithm.   If this is not true, choose
$$
n_{i+1}=1+ (n_i-1)\max\left\{2,\left\lceil\frac{1}{(n_i-1)}\sqrt{\frac{\tau \tF_{n_i}(f)}{8\varepsilon}}\right\rceil\right\}.
$$
Go to Stage 1.
\end{description}
\end{algo}


\Section{Simpson's Rule}

\begin{algo}[Adaptive Univariate Integration] \label{multistageintegalgosimpson}
Given an interval $[a,b]$, an inflation function, $\mathfrak{C}$, a positive key mesh size $\mathfrak{h}$, a positive error tolerance, $\varepsilon$, and a routine for generating values of the integrand, $f$, set $l=1$, and
$n_1=2(\lfloor (b-a)/\mathfrak{h}\rfloor+1)$.%, if $\lfloor(n_1/2)\rfloor\neq n_1/2$, $n_1=n_1+1$.
\begin{description}
\item[Stage 1] %Estimate {$\|f''-4[f(1)-2f(1/2)+f(0)]\|_1$} and bound {$\Var(f''')$}.]
Compute the error estimate $\widetilde{\text{err}}(f,n_l)$ according to \eqref{errorboundcone}.

\item[Stage 2] %Check the necessary condition for $f \in \cc_{\tau}$.]
If $\widetilde{\text{err}}(f,n_l)\le\varepsilon$, then return the Simpson's rule approximation $S_{n_l}(f)$ as the answer.

\item[Stage 3] %Check for convergence.]
Otherwise let $n_{l+1}=\max(2,m)\eta_{l}$, where
$$m=\min\{r\in\mathbb{N}:\eta(rn_l)\widetilde{V}_{n_{l}(f)}\le\varepsilon\}, \text{with } \eta(n):=\frac{(b-a)^4\mathfrak{C}(2(b-a)/n)}{5832n^4}.$$
increase $l$ by one, and go to 1.
\end{description}
\end{algo}

\begin{theorem}\label{thmSimpson}
    Algorithm \ref{multistageintegalgosimpson} is successful, i.e.,
    \begin{equation*}
      \left|\int_{a}^{b}f(x)dx-\texttt{integral}(f,a,b,\varepsilon)\right|\le\varepsilon, \qquad \forall f\in \cc.
    \end{equation*}
\end{theorem}


\Chapter{Numerical Experiments}

\Section{Trapezoidal temp}
Consider the family of bump test functions defined by
\begin{multline}\label{testfun}
f(x)= \\
\begin{cases}
\displaystyle  b[4a^2 + (x-z)^2 + (x-z-a)|x-z-a|\\
\qquad \qquad -(x-z+a)|x-z+a|], & z-2a\leq x\leq z+2a,\\[2ex]
\displaystyle  0, & \text{otherwise}.
\end{cases}
\end{multline}
with  $\log_{10}(a) \sim \cu[-4,-1]$, $z \sim \cu[2a,1-2a]$, and $b=1/(4a^3)$ chosen to make $\int_0^1 f(x) \, \dif x = 1$.  It follows that $\norm[1]{f'-f(1)+f(0)}=1/a$ and $\Var(f')=2/a^2$.  The probability that $f \in \cc_{\tau}$ is $\min\left(1,\max(0,\left(\log_{10}(\tau/2)-1\right)/3)\right).$

As an experiment, we chose $10000$ random test functions and applied Algorithm \ref{multistageintegalgo} with an error tolerance of  $\varepsilon = 10^{-8}$ and initial $\tau$ values of $10, 100, 1000$.  The algorithm is considered successful for a particular $f$ if the exact and approximate integrals agree to within $\varepsilon$. The success and failure rates are given in Table \ref{integresultstable}. Our algorithm imposes a cost budget of $N_{\max}=10^7$.  If the proposed $n_{i+1}$ in Stages 2 or 3 exceeds $N_{\max}$, then our algorithm returns a warning and falls back to the largest possible $n_{i+1}$ not exceeding $N_{\max}$ for which $n_{i+1}-1$ is a multiple of $n_i-1$.  The probability that $f$ initially lies in $\cc_{\tau}$ is the smaller number in the third column of Table \ref{integresultstable}, while the larger number is the empirical probability that $f$ eventually lies in $\cc_{\tau}$ after possible increases in $\tau$ made by Stage 2 of Algorithm \ref{multistageintegalgo}.  For this experiment Algorithm \ref{multistageintegalgo} was successful for all $f$ that finally lie inside $\cc_{\tau}$ and for which no attempt was made to exceed the cost budget.

\begin{table}[h]
\centering
\begin{tabular}{cccccc}
&&&Success & Success & Failure \\
& $\tau$ &  $\Prob(f \in \cc_{\tau}) $ & No Warning & Warning & No Warning \\
\toprule
&$10$ & $0\% \rightarrow  25\% $ & $25\%$ & $<1\%$ & $75\%$  \\
Algorithm \ref{multistageintegalgo}
 &$100$ & $23 \% \rightarrow 58\% $ & $56\%$ & $2\%$ & $42\%$ \\
&$1000$ & $57\% \rightarrow 88\% $& $68\%$ & $20\%$ &$12\%$ \\
\midrule
{\tt quad} & & & 8\% & & $92\%$\\
{\tt integral} & & & 19\% & & $81\%$\\
{\tt chebfun} & & &29\% & & $71\%$\\
\end{tabular}
\caption{The probability of the test function lying in the cone for the original and eventual values of $\tau$ and the empirical success rate of Algorithm \ref{multistageintegalgo} plus the success rates of other common quadrature algorithms. \label{integresultstable}}
\end{table}

Some commonly available numerical algorithms in MATLAB are {\tt quad} and {\tt integral} \cite{MAT8.1} and the MATLAB Chebfun toolbox \cite{TrefEtal12}. We applied these three routines to the random family of test functions.  Their success and failure rates are also recorded in Table \ref{integresultstable}.  They do not give warnings of possible failure.

\Section{Simpson's Rule}
\textcolor{red}{I need to use a good example to run the tests. Then I will need to compare the results for both algorithms with other algorithms. After that, I need to compare trap and sim to get the conclusion such as sim has faster convergence rate than trap. This will require an sample function good to both trap and sim.}


\Chapter{CONCLUSION}
 %   \input{Conclusion.tex}

A


\Section{Summary}

A

\clearpage


%
% APPENDIX
%

% Do the settings of appendices with \appendix command
\appendix

% Then create each appendix using
% \Appendix{title_of_appendix} command

\Appendix{Table of Transition Coefficients for the Design of
Linear-Phase FIR Filters}

Your Appendix will go here !

% \moretox

  \Appendix{Name of your Second
Appendix}

Your second appendix text....

\Appendix{Name of your Third Appendix}

Your third appendix text....
%
% BIBLIOGRAPHY
%
% you have two options: 1) create bibliography manually,
% 2) create bibliography automatically. See BibliographyHelp.pdf file for details.

%
%\bibliographystyle{plain}
%\bibliography{mybib}

\end{document}  % end of document
