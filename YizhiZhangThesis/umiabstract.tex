% $Id: umiabstract.tex,v 1.2 2003/12/30 02:16:08 norman Exp $
%
This dissertation describes the measurement of the decay of the long lived neutral kaon into two muons and two electrons.  The measurement was performed using the data taken during experiment E871 which was performed on the B5 beamline at the Alternating Gradient Synchrotron (AGS) of the Brookhaven National Laboratory (BNL).  The branching ratio B(\KtoMuMuEE) is sensitive to the absorptive portion of the long distance amplitude for decays of the form $\Klong \rightarrow \TwoLepton$ and can be used to properly extract the short distance weak interaction amplitudes from the dileptonic events.

Measurement of \KtoMuMuEE additionally allows for the exploration of the form factor for the \Kgamstargamstar vertex.  Measurement of the \KtoMuMuEE branching fraction from the E871 data set provides a sensitive probe to distinguish between form factors arising from a chiral theory near the kaon mass, a low energy quark/QCD theory, a vector meson dominance model, models with CP violation and models with exhibiting a uniform phasespace.

The analysis of the data from the E871 $\mu\mu$ data stream observed 119 \KtoMuMuEE events on a measured background of 52 events.  The \KtoMuMuEE event sample was normalized using simultaneously measured sample of 5685 \KtoMuMu events.  The resulting branching fraction for \KtoMuMuEE was calculated to be $2.78\pm0.41\pm0.09\times 10^{-9}$ under the assumption of a $\chi$PT form factor.  The results are consistent with the world average for B(\KtoMuMuEE) and increase the total number of \KtoMuMuEE events observed world wide from 152 to 271. 


%This dissertation describes the measurement of the decay of the long lived neutral kaon into two muons and two electrons.  The measurement was performed using the data taken during experiment E871 which was performed on the B5 beamline at the Alternating Gradient Synchrotron (AGS) of the Brookhaven National Laboratory (BNL).  The branching ratio B(\KtoMuMuEE) is sensitive to the absorptive portion of the long distance amplitude for decays of the form $\Klong \rightarrow \TwoLepton$ and can be used to properly extract the short distance weak interaction amplitudes from the dileptonic events.  Measurement of \KtoMuMuEE additionally allows for the exploration of the form factor for the \Kgamstargamstar vertex which provides a sensitive probe for chiral theories near the kaon mass.  This data set provides ???? events at a measured branching fraction of ????.  The results are consistent with the world average and improve upon the observed number events by ????.

%%% Local Variables: 
%%% mode: plain-tex
%%% TeX-master: "~/Thesis/Thesis_Draft"
%%% End: 
