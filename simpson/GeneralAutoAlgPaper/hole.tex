\documentclass[]{elsarticle}
%\setlength{\marginparwidth}{0.5in}
\usepackage{amsmath,amssymb,amsthm,mathtools,bbm,booktabs,array,tikz,pifont,comment,multirow,url,graphicx}
\input FJHDef.tex

%Requires ApproxUnivariate_k.tex, univariate_integration_k.tex, ConesPaperSpikyquad.eps, ConesPaperFlukyquad.eps

\DeclareMathOperator{\Var}{Var}
\DeclareMathOperator{\INT}{INT}
\DeclareMathOperator{\APP}{APP}
\DeclareMathOperator{\lin}{lin}
\DeclareMathOperator{\up}{up}
\DeclareMathOperator{\lo}{lo}
\DeclareMathOperator{\fix}{fix}
\DeclareMathOperator{\err}{err}
\DeclareMathOperator{\maxcost}{maxcost}
\DeclareMathOperator{\mincost}{mincost}
\newcommand{\herr}{\widehat{\err}}

\newtheorem{theorem}{Theorem}
\newtheorem{prop}[theorem]{Proposition}
\newtheorem{lem}{Lemma}
\newtheorem{cor}{Corollary}
\theoremstyle{definition}
\newtheorem{algo}{Algorithm}
\newtheorem{condit}{Condition}
%\newtheorem{assump}{Assumption}
\theoremstyle{remark}
\newtheorem{rem}{Remark}
\newcommand{\Fnorm}[1]{\abs{#1}_{\cf}}
\newcommand{\Ftnorm}[1]{\abs{#1}_{\tcf}}
\newcommand{\Gnorm}[1]{\norm[\cg]{#1}}
\newcommand{\flin}{f_{\text{\rm{lin}}}}

\journal{Journal of Complexity}

\begin{document}

What we have known:

The stronger semi-norm is $\Fnorm{f}:=\Var(f''')$.

The weaker semi-norm is
\[
\Ftnorm{f}:=\norm[1]{f''-4[f(1)-2f(1/2)+f(0)]}.
\]

The cone of integrands is
\begin{equation}\label{coneinteg}
\cc_{\tau}:=\left\{f\in \cv^{3}:\begin{cases}
\displaystyle \Var(f''')\leq\tau\Var(f'')\\[2ex]
\displaystyle \Var(f'')\leq\tau\|f''-4[f(1)-2f(1/2)+f(0)]\|_1
\end{cases}\right\}.
\end{equation}

The algorithm for approximating $\norm[1]{f''-4[f(1)-2f(1/2)+f(0)]}$ is:
\begin{align}
\nonumber
\tF_n(f)&:=\Ftnorm{A_n(f)}=\bignorm[1]{A_n(f)''-A_3(f)''} \\
\label{1direst} \nonumber
&=\sum_{j=1}^{(n-1)/2}\left|2(n-1)(f(x_{2j+1})-2f(x_{2j})+f(x_{2j-1})) - \frac{8}{n-1}[f(1)-2f(1/2)+f(0)]\right|.
\end{align}
And,
\begin{equation} \label{Fnormalg}
F_n(f) :=\Var(A_n(f)''') = (n-1)^2\sum_{j=1}^{(n-3)/2} \bigabs{f(x_{2j+3}) - 2 f(x_{2j+2})+2f(x_{2j})-f(x_{2j-1})}.
\end{equation}

What we want to prove:
\begin{align*}
&\Ftnorm{f} \ge \tF_n(f),\\
&\Ftnorm{f} - \tF_n(f)  \ge 0.
\end{align*}
And
\begin{align*}
\Ftnorm{f} - \tF_{n}(f) & \le \bignorm[1]{f'' -A_n(f)''}\\
& \le \frac{\Var(f'')}{4(n-1)},\\
& \le \frac{\tau}{2(n-1)}\Ftnorm{f}.
\end{align*}
So
\begin{align*}
  \Ftnorm{f} \le \frac{2n-2}{2n-2-\tau}\tF_n(f).
\end{align*}

We also know that the error of the Simpson's rule in terms of the variation of the first derivative of the integrand is:
%\begin{subequations} \label{integhhtilde}
\begin{gather*}
\abs{\int_0^1 f(x) \, dx - P_n(f)} \le \frac{1}{72(n-1)^4} \Var(f''').\\
\end{gather*}
%\end{subequations}

So the approximation error of Simpson's rule should be:
\begin{align*}
&\abs{\int_0^1 f(x) \, dx - P_n(f)},\\
 \le &\frac{1}{72(n-1)^4} \Var(f'''),\\
 \le &\frac{1}{72(n-1)^4} \tau\Var(f''),\\
 \le &\frac{1}{72(n-1)^4} \tau^2\Ftnorm{f},\\
 \le &\frac{1}{72(n-1)^4} \tau^2\frac{2n-2}{2n-2-\tau}\tF_n(f),\\
 = & \frac{\tau^2 \tF_n(f)}{36(n-1)^3(2n-2-\tau)}.
\end{align*}

Consider this example: let $g(x)=f-A_n(f)$. Suppose
\begin{align*}
  g''(x)= \begin{cases} \displaystyle a_{-}, & 0 \le x < \xi,\\[1ex]
\displaystyle  a_{+}, & \xi \le x \le 1,
\end{cases}
\end{align*}
Then
\begin{align*}
  g'(x)= \begin{cases} \displaystyle a_{-}(x-\xi)+b, & 0 \le x < \xi,\\[1ex]
\displaystyle  a_{+}(x-\xi)+b, & \xi \le x \le 1,
\end{cases}
\end{align*}
\begin{align*}
  g(x)&= \begin{cases} \displaystyle \frac{1}{2}a_{-}[(x-\xi)^2-\xi^2]+bx, & 0 \le x < \xi,\\[1ex]
\displaystyle  \frac{1}{2}a_{+}[(x-\xi)^2-(1-\xi)^2]-b(1-x), & \xi \le x \le 1,
\end{cases}\\
&=\begin{cases} \displaystyle \frac{1}{2}a_{-}[x(x-2\xi)]+bx, & 0 \le x < \xi,\\[1ex]
\displaystyle  \frac{1}{2}a_{+}[(x-1)(x+1-2\xi)]-b(1-x), & \xi \le x \le 1,
\end{cases}
\end{align*}




\end{document}

